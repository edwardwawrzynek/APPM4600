\documentclass[10pt]{article}
\usepackage[utf8]{inputenc}
\usepackage[includehead, headheight=10mm, margin=15mm ]{geometry}
\usepackage{amsmath}
\usepackage{amsthm}
\usepackage{amsfonts}
\usepackage{xcolor}
\usepackage{graphicx}
\usepackage{titling}
\usepackage{fancyhdr}
\usepackage{listings}

\title{APPM 4600 Homework 2}
\author{Edward Wawrzynek}
\date{14 September 2024}

\newcommand*{\dif}{\mathop{}\!\mathrm{d}}

\makeatletter
\def\@maketitle{%
  \newpage
  \null
  \vskip 1em%
  \begin{center}%
  \let \footnote \thanks
    {\LARGE \@title \par}%
    \vskip 1em%
    {\normalfont \@date}
  \end{center}%
  \par
  \vskip 1em}
\makeatother

\begin{document}

\pagestyle{fancy}
    \fancyhf{} % clear all header and footer fields
    \fancyhead[L]{\thetitle}
    \fancyhead[R]{\theauthor}

\makeatletter
\begin{center}
    {\Large \@title}
    \vskip 1mm
    {\normalfont \@date}
    \vskip 1em
\end{center}
\makeatother

\begin{enumerate}
    \item \begin{enumerate}
      \item Consider the function \begin{align*}
          f(x) = (1+x)^n - 1 - nx.
      \end{align*}Observe that \begin{align*}
        \lim_{x \to 0} \left|\frac{f(x)}{x}\right| = \lim_{x \to 0} \left|\frac{(1+x)^n - 1 - nx}{x}\right| = \lim _{x\to 0} \left| n(1+x)^{n-1}-n \right| = 0.
      \end{align*}
      Thus, \(f(x) = o(x)\) as \(x \to 0\), which implies that \begin{align*}
          (1+x)^n = 1 + nx + o(x).
      \end{align*}

      \item Observe that \begin{align*}
          \lim _{x\to 0} \left| \frac{x\sin \sqrt{x}}{x^\frac{3}{2}} \right| = \lim _{x\to 0} \left| \frac{\sin \sqrt{x} + \frac{1}{2}\sqrt{x}\cos \sqrt{x}}{\frac{3}{2}\sqrt{x}} \right| = \lim _{x\to 0} \left| \frac{\frac{3}{4\sqrt{x}}\cos \sqrt{x} - \frac{1}{4}\sin \sqrt{x}}{\frac{3}{4\sqrt{x}}} \right| &= \lim _{x\to 0} \left| \cos \sqrt{x} - \frac{\sqrt{x}}{3}\sin \sqrt{x} \right| \\ &= 1 \neq 0.
      \end{align*} Thus, as \(x \to \infty\), \begin{align*}
          x\sin  \sqrt{x} = O\left(x^\frac{3}{2}\right).
      \end{align*}

      \item Observe that \begin{align*}
          \lim _{t\to\infty} \left| \frac{e^{-t}}{\frac{1}{t^2}} \right| = \lim _{t\to\infty} \left| \frac{t^2}{e^t} \right| = \lim _{t\to\infty} \left| \frac{2t}{e^t} \right| = \lim _{t\to\infty} \left| \frac{2}{e^t} \right| = 0.
      \end{align*} Thus, as \(t \to \infty\), \begin{align*}
          e^{-t} = o\left(\frac{1}{t^2}\right).
      \end{align*}

      \item Observe that \begin{align*}
        \lim _{\epsilon \to 0} \left| \frac{\int_0^\epsilon e^{-x^2}\dif x}{\epsilon } \right| = \lim _{\epsilon \to 0} \left| \frac{\dif}{\dif x}\int_0^\epsilon e^{-x^2}\dif x \right| = \lim _{\epsilon \to 0} \left|  e^{-\epsilon^2} \right| = 1 \neq 0.
      \end{align*} Thus, as \(\epsilon  \to 0\), \begin{align*}
        \int_0^\epsilon e^{-x^2}\dif x = O(\epsilon).
      \end{align*}
    \end{enumerate}

    \item \begin{enumerate}
      \item We have the exact problem \(A\vec{x}=\vec{b}\) with solution \(\vec{x} = A^{-1}\vec{b}\). The perturbed problem is \(A\vec{x^\ast}=(\vec{b}+\vec{\Delta b})\) with solution \begin{align*}
          \vec{x^\ast} = A^{-1}(\vec{b}+\vec{\Delta b}).
      \end{align*} The change in solution is \begin{align}
        \vec{\Delta x} = \vec{x^\ast} - \vec{x} = A^{-1} \left(\vec{b} +  \vec{\Delta b} - \vec{b}\right) &= A^{-1} \vec{\Delta b} \nonumber \\ 
        &= \begin{bmatrix}
          1-10^{10} & 10^{10} \\ 1 + 10^{10} & -10^{10}
        \end{bmatrix} \begin{bmatrix}
          \Delta b_1 \\ \Delta b_2
        \end{bmatrix} \nonumber \\
        &= \begin{bmatrix}
          \Delta b_1 + 10^{10} \left( \Delta b_2 - \Delta b_1 \right) \\
          \Delta b_1 + 10^{10} \left( \Delta b_1 - \Delta b_2 \right)
        \end{bmatrix}\cdot  \label{eq:2pert}
      \end{align}

      \item The condition number of \(A\) is bounded as \[\kappa(A) \leq \frac{\sigma_1}{\sigma_n},\] where \(\sigma_1\) and \(\sigma_2\) are the singular values of \(A\). We have that A has singular value decomposition \begin{align*}
          A = \begin{bmatrix}
            10^{-10} & 1 \\
            -1       & 10^{-10}
          \end{bmatrix} \begin{bmatrix}
            5\times_10^9 & 0 \\ 0 & \frac{1}{2}
          \end{bmatrix} \begin{bmatrix}
            0 & 1 \\ 1 & 10^{-10}
          \end{bmatrix},
      \end{align*} so we have \(\sigma_1 = 5\times 10^9\) and \(\sigma_2 = \frac{1}{2}\) and the condition number is bound as \begin{align*}
          \kappa (A) \leq \frac{\sigma_1}{\sigma_2} = \frac{5\times_10^9}{\frac{1}{2}} = 10^10,
      \end{align*} which suggests that the problem may be poorly conditioned.

      \item The relative error in the solution is \(||\vec{\Delta x}|| / ||\vec{x}||\). The condition number provides an upper bound for the relative error in the solution relative to the perturbation.
      
      From the absolute error \eqref{eq:2pert}, we known that the error is dominated by the \(\Delta b_1 - \Delta b_2\) term. The error will be larger if the perturbations in each dimension are not matched. The relative error for various perturbations is given below.

      \begin{center}
        \begin{tabular}{c|c}
          Perturbation & Error \\
          \hline
          \(\Delta b_1 = 10^{-5}, \Delta b_2 = 10^{-5}\) & \(1.002\times 10^{-5}\) 
          \\ \hline
          \(\Delta b_1 = 2\times10^{-5}, \Delta b_2 = 0.7\times10^{-5}\) & \(1.3 \times 10^5\)
        \end{tabular}
      \end{center}

      In general, we cannot expect the perturbation terms to be equal, and we observe a very poorly conditioned system.
    \end{enumerate}

    \item \begin{enumerate}
      \item  
    \end{enumerate}

    \section*{A. Codes}
    The code used to compute the results in the table in 2(c) is listed below.
    {\small \lstinputlisting[language=Python]{hw2_2c.py}}

\end{enumerate}


\end{document}
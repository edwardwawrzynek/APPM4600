\documentclass[10pt]{article}
\usepackage[utf8]{inputenc}
\usepackage[includehead, headheight=10mm, margin=15mm ]{geometry}
\usepackage{amsmath}
\usepackage{amsthm}
\usepackage{amsfonts}
\usepackage{xcolor}
\usepackage{graphicx}
\usepackage{titling}
\usepackage{fancyhdr}
\usepackage{listings}
\usepackage{hyperref}
\usepackage{listings}


\title{APPM 4600 Lab 6}
\author{Edward Wawrzynek}
\date{3 October 2024}

\newcommand*{\dif}{\mathop{}\!\mathrm{d}}

\makeatletter
\def\@maketitle{%
  \newpage
  \null
  \vskip 1em%
  \begin{center}%
  \let \footnote \thanks
    {\LARGE \@title \par}%
    \vskip 1em%
    {\normalfont \@date}
  \end{center}%
  \par
  \vskip 1em}
\makeatother

\begin{document}

\pagestyle{fancy}
    \fancyhf{} % clear all header and footer fields
    \fancyhead[L]{\thetitle}
    \fancyhead[R]{\theauthor}

\makeatletter
\begin{center}
    {\Large \@title}
    \vskip 1mm
    {\normalfont \@date}
    \vskip 1em
\end{center}
\makeatother

The code for this lab can be seen on github \href{https://github.com/edwardwawrzynek/APPM4600/blob/master/Labs/Lab\%206/lab6.py}{here} and is included below.

\section{Prelab: Finite Differences}
\begin{enumerate}
  \item The implementation of the forward and central difference is given in the code below. The approximate derivatives for \(f(x) = \cos  x\) at \(x = \pi /2\) are listed below in Table \ref{tab:fd} and \ref{tab:cd} below.
  
  \begin{table}[h!]
    \centering
    \begin{tabular}{c c} 
     \(h\) & Forward Difference at \(x = \frac{\pi}{2}\) \\
     \hline
     0.01 & -0.9999833334166673 \\
0.005 & -0.9999958333385205 \\
0.0025 & -0.9999989583336375 \\
0.00125 & -0.9999997395833322 \\
0.000625 & -0.999999934895991 \\
0.0003125 & -0.9999999837237595 \\
0.00015625 & -0.9999999959315011 \\
7.8125e-05 & -0.9999999989818379 \\
3.90625e-05 & -0.9999999997447773 \\
1.953125e-05 & -0.9999999999355124 \\
    \end{tabular}
    \caption{Forward differences of \(f(x = \frac{\pi}{2})\).}
    \label{tab:fd}
    \end{table}

    \begin{table}[h!]
      \centering
      \begin{tabular}{c c} 
       \(h\) & Central Difference at \(x = \frac{\pi}{2}\) \\
       \hline
       0.01 & -0.9999833334166673 \\
       0.005 & -0.9999958333385205 \\
       0.0025 & -0.9999989583336376 \\
       0.00125 & -0.9999997395833324 \\
       0.000625 & -0.9999999348959908 \\
       0.0003125 & -0.9999999837237594 \\
       0.00015625 & -0.9999999959315011 \\
       7.8125e-05 & -0.9999999989818379 \\
       3.90625e-05 & -0.9999999997447773 \\
       1.953125e-05 & -0.9999999999355121 \\
      \end{tabular}
      \caption{Central differences of \(f(x = \frac{\pi}{2})\).}
      \label{tab:cd}
      \end{table}
  
      \item Both methods have numerical order \(\alpha  \approx 1\), as expected.
  
\end{enumerate}

\section{Slacker Newton}
We choose to update the Jacobian whenever the current point is sufficiently far from the last point at which we computed the Jacobian, that is, where \begin{align*}
  ||x_0 - x_n|| > t,
\end{align*} where \(t\) is some tolerance. The implementation is provided in the code at the end of the document.

We choose \(t = 10^{-5}\) and find the approximate root \begin{align*}
    x \approx 0.99860694, \; y \approx -0.10553049.
\end{align*} This is the same as is found by lazy newton, except that slacker newton takes 3 iterations to reach \(10^{-10}\) tolerance whereas lazy newton takes 7 iterations.

\section{Approximate Jacobian}
We approximate the Jacobian with forward differences, that is, \begin{align*}
    J(x,y) \approx \begin{bmatrix}
      \frac{f(x+h,y)-f(x,y)}{h} & \frac{f(x, y+h)-f(x, y)}{h} \\
      \frac{g(x+h,y)-g(x,y)}{h} & \frac{g(x, y+h)-g(x, y)}{h}.
    \end{bmatrix}
\end{align*}

The implementation is provided in the code at the end of the document.

We pick \(h = 10^{-8}\) and find the approximate root \begin{align*}
  x \approx 0.99860694, \; y \approx -0.10553049,
\end{align*} which is found in 3 iterations.

{\small \lstinputlisting[language=Python]{lab6.py}}   

\end{document}